%TCIDATA{Version=5.00.0.2570}
%TCIDATA{LaTeXparent=0,0,sw-edit.tex}

% $Id: acronyms.tex 1735 2017-06-23 15:47:39Z lv39 $
% $URL: https://forge.cornell.edu/svn/repos/ldi-replication/LaTeX/text/acronyms.tex $
%
% Define acronyms to be used in the text here. See
% http://www.mackichan.com/index.html?techtalk/456.htm~mainFrame for usage in
% Scientific workplace context

\begin{acronym}
\acro{ACS}{American Community Survey}
\acro{AEA}{American Economic Association}
\acro{AEJ:AE}{American Economic Journal: Applied Economics}
\acro{AEJ:Mac}{American Economic Journal: Macroeconomics}
\acro{AEJ:Mic}{American Economic Journal: Microeconomics}
\acro{AEJ:EP}{American Economic Journal: Economic Policy}
\acro{AER}{American Economic Review}
\acro{AJPS}{American Journal of Politial Science}
\acro{AHEAD}{Study of Assets and Health Dynamics Amongst the Oldest Old}
\acro{API}{application programming interface}
\acro{ASCII}{American Standard Code for Information  Interchange} %, typically used to denote raw text files in PC or Unix environments
\acro{ASM}{Annual Survey of Manufacturers}
\acro{BDS}{Business Dynamics Statistics}
\acro{BED}{Business Employment Dynamics}
\acro{BES}{Business Expenditure Survey}
\acro{BLS}{Bureau of Labor Statistics}
\acro{BRB}{Business Register Bridge}
\acro{BR}{Business Register}
\acro{CAC}{Cornell Center for Advanced Computing}
\acro{CBP}{County Business Patterns}
\acro{CBSA}{Core-Based Statistical Area}
\acro{CER}{Covered Earnings Records}
\acro{CES}{Center for Economic Studies}
\acro{CEW}{Covered Employment and Wages}%. Employment statistics program run by BLS in  conjunction with all states, also known as ES-202. Generally, when used  in this document, refers to public-use tabulations from the CEW, as  opposed to the confidential microdata received directly from the states.
\acro{CISER}{Cornell Institute for Social and Economic Research}
\acro{CIT}{Cornell Information Technologies}
\acro{CODA}{Children of Depression}
\acro{CPI}{Consumer Price Index}
\acro{CPI-U}{Consumer Price Index (All Urban Consumers)}
\acro{CPR}{Composite Person Record}
\acro{CPS}{Current Population Survey}
\acro{CRADC}{Cornell Restricted Access Data Center}
\acro{CTC}{Cornell Theory Center}
\acro{DCC}{Data Confidentiality Committee}
\acro{DOI}{Digital Object Identifier}
\acro{DER}{Detailed Earnings Record}
\acro{DRB}{Disclosure Review Board}
\acro{DWS}{Displaced Worker Supplement}
\acro{EJ}{Economic Journal}
\acro{ECF}{Employer Characteristics  File}
\acro{EHF}{Employment History Files}
%\acro{EIN}{\acroextra{(federal) }Employer Identification Number}
\acro{ERR}{Excess Reallocation Rate}
%\acro{ES-202}{ES-202\acroextra{. An older name for the \ac{QCEW} program}}
\acro{FHFA}{Federal Housing Finance Agency}
%\acro{FIPS}{Federal information processing standards codes\acroextra{\ issued     by \ac{NIST}}}
\acro{FSRDC}{Federal Statistical Research Data Center}
%\acro{FTI}{Federal Tax Information\acroextra{, typically covered under     Title 26, U.S.C.}}
\acro{GAL}{Geocoded Address List}
\acro{GIS}{Geographic Information System}
\acro{HPI}{House Price Index}
\acro{HRS}{Health and Retirement Study}
\acro{IAB}{Institute for Employment Research}
\acro{ICF}{Individual Characteristics File}
\acro{IRB}{Institutional Review Board}
\acro{IRS}{Internal Revenue Service}
\acro{ISR}{Institute for Social Research}
\acro{JCR}{Job Creation Rate}
\acro{JDR}{Job Destruction Rate}
\acro{JOLTS}{Job Openings and Labor Turnover Survey}
\acro{JASA}{Journal of the American Statistical Association}
\acro{JMCB}{Journal of Money, Credit and Banking}
\acro{JPE}{Journal of Political Economy}
\acro{JEEA}{Journal of the European Economic Association}
\acro{JRR}{Job Reallocation Rate}
\acro{LAUS}{Local Area Unemployment Statistics}
\acro{LBD}{Longitudinal Business Database}
%\acro{LDB}{\ac{BLS}'s Longitudinal Business Database}
\acro{LED}{Local Employment Dynamics}
\acro{LEHD}{Longitudinal Employer-Household Dynamics}
\acro{LMI}{Labor Market Information}
\acro{MBR}{Master Beneficiary Record}
\acro{MEF}{Master Earnings File}
\acro{MER}{Master Earnings Record}
\acro{MLS}{Mass Layoff Statistics}
\acro{MMS}{Methodology, Measurement, and Statistics}
\acro{MN}{Minnesota}
\acro{MSA}{Metropolitan Statistical Area}
\acro{MSD}{Metropolitan Statistical Division}
\acro{MWR}{Multiple Worksite Report}
\acro{NAICS}{North American Industry Coding System}
\acro{NECTA}{New England  City and Town Area}
\acro{NIA}{National Institute on Aging}
\acro{NIST}{National Institute of Standards and Technology}
\acro{NLSY}{National Longitudinal Study of Youth}
\acro{NSF}{National Science Foundation}
\acro{NSTA}{NAICS SIC Treatment of Auxiliaries}
\acro{OS}{operating system}
\acro{OTM}{OnTheMap}
\acro{PCF}{Person Characteristics File}
\acro{PHF}{Person History File}
\acro{PIK}{Protected Identity Key}
\acro{PSID}{Panel Study of Income Dynamics}
%\acro{QCEW}{Quarterly Census of Employment and Wages\acroextra{, managed by   the \acf{BLS}}}
\acro{QJE}{Quarterly Journal of Economics}
\acro{QWI}{Quarterly Workforce Indicators}
\acro{RDA}{Restricted Data Application}
\acro{RDC}{Research Data Center}
\acro{ReStat}{Review of Economics and Statistics}
\acro{RUN}{Reporting unit number}
%\acro{SEIN}{State employer identification number\acroextra{. It is     constructed from the state \ac{FIPS} code and the UI account     number. The BLS refers to the UI account number in combination with the     reporting unit number as SESA-ID}}
\acro{SEINUNIT}{SEIN reporting unit}
\acro{SEPB}{Summary of Earnings and Projected Benefits} % confidential SSA                                % file
%\acro{SESA-ID}{State Employment Security Agency ID\acroextra{. The UI     account number in combination with the Reporting Unit Number is treated   as a unique establishment identifier.}}
\acro{SESA}{State Employment Security Agency}
\acro{SIC}{Standard Industry Classification}
\acro{SIPP}{Survey of Income and Program Participation}
\acro{SLID}{Survey of Labour and Income Dynamics}
\acro{SPF}{Successor-Predecessor File}
\acro{SRMI}{Sequential Regression Multiple Imputation}
\acro{SSA}{Social Security Administration}
\acro{SSI}{Supplemental Security Income}
\acro{SSN}{Social Security Number}
\acro{SSR}{Supplemental Security Record}
%\acro{SynLBD}{Synthetic \ac{LBD}\acroextra{, a synthetic microdata file at the establishment level}}
\acro{U2W}{Unit-to-Worker Impute}
\acro{UI}{Unemployment Insurance}
\acro{URL}{Uniform Record Locator}
\acro{WB}{War Babies}
\acro{WIA}{Workforce Investment Act}
\acro{WIB}{Workforce Investment Board}
\acro{WRR}{Worker Reallocation Rate}
\acro{WTS}{Windows Terminal Services}
\acro{VCS}{version control system}
% Usage in the later text:
%  \ac{acronym}         Expand and identify the acronym the first time; use
%                       only the acronym thereafter
%  \acf{acronym}        Use the full name of the acronym.
%  \acs{acronym}        Use the acronym, even before the first corresponding
%                       \ac command
%  \acl{acronym}        Expand the acronym without using the acronym itself.
\end{acronym}

%%% Local Variables:
%%% mode: latex
%%% TeX-master: "proposal"
%%% End:
