

\noindent
Journals have pushed for transparency of research through data availability policies. Such data policies improve  availability of data and code, but what is the impact on reproducibility? We present results from a large reproduction exercise for  articles published in the American Economic Journal: Applied Economics, which has had a data availability policy since its inception in  2009. 
%
Out of \input{includes/aejdoistotal} published articles, we assessed  \input{includes/total.assessed} articles. All articles provided some materials. We excluded \input{./includes/total.excluded} articles that required confidential or proprietary data, or that required the replicator to otherwise obtain the data (\input{./includes/excluderate}\% of assessed articles). 
%
We attempted to reproduce \input{includes/total.attempted.nonconf} articles, and were able to fully reproduce the results of \input{includes/attempted.success}  (\input{includes/successrate}\% of attempted reproductions). A further \input{includes/attempted.partial} (\input{includes/partialrate}\% of attempted reproductions) were partially reproduced. Many articles required complex code changes even when at least partially reproduced. We collect bibliometric characteristics of authors, but find no  evidence for author  characteristics as determinants of reproducibility. There does not appear to be a citation bonus for reproducibility.

The data availability policy of this journal was effective to ensure availability of materials, but is insufficient to ensure reproduction without additional work by replicators. 