% Compare OpenAlex and Web of Science

\subsection{Web of Science}

In an earlier version of this paper, we had used data from Web of Science \parencite[WoS]{WebofScience}.
We manually queried the WoS  database citations for each article by entering the partial DOI of articles in an issue of AEJ:AE. For later issues, the DOI structure changed, and alternate search criteria were used. For each search results, WoS provided year-by-year citations, as well as total and average citations. We did this in 2015, 2017, and after the conclusion of the exercise, in 2019.

For each of up to five authors per article, we also (contemporaneously) queried WoS, searching for that author, and recording their h-index \parencite{Hirsch2005} and the underlying number of citations for each author by year, as well as the search criteria used to find the author. In some cases, a simple search by author name does not yield a unique person (e.g.,  ``Smith, Adam''), and sometimes, the metadata in Web of Science contained errors.
%\footnote{For example, we only found one article for ``Lawrence E. Katz'' in Web of Science as of January 2016, namely the article in \ac{AEJ:AE} \parencite{10.1257/app.2.3.228}, but did find quite a few more for ``Lawrence F. Katz.'' While we initially thought this to be the result of some inside joke for senior economists, even the \ac{AEJ:AE} website displays the author of the article (correctly) as ``Lawrence F. Katz,'' and we have no explanation for how this error could persist in the Web of Science database (the error is still present in 2023). Our search criteria adjusted for this error.} 

We adjusted total citations as reported for years since publication (which differed for each issue), and used that adjusted number in our analysis. Results in \textcite{kingi2018,herbert2021} relied on these numbers. By design, we excluded several years worth of articles from these analyses, since not enough time had elapsed since the article had been published.

\subsection{OpenAlex}

In response to a referee, we investigated expanding the measurement again, for a larger set of articles over a larger expanse of time. The very manual process combined with an absence of research participants and authors working on this project suggested an alternative approach, which had become feasible in the interim, thanks to the efforts of \textcite{openalex2022}. We therefore queried the OpenAlex (OA) corpus \parencite{ourresearch2023} via the openly accessible \ac{API}. Code to do so is included in the replication package. 

Generically, OA has more works than WoS. To facilitate the comparison, when computing as-of-year h-index, we remove a few types of works that are not usually encountered on WoS or for that matter on Google Scholar (GS): "\texttt{other}", "\texttt{paratext}", "\texttt{peer-review}", "\texttt{reference-entry}". 

For the \input{./includes/totalauthorsOA} authors in our sample, we identified \input{./includes/ninstitutionsOA}. The median institution is recorded as having published \input{./includes/median_inst.works.tex} as of 2023. Cornell University, an R1 research university, is recorded as having published \input{./includes/cornell.works}, whereas Wellesley College, a liberal arts college focused on teaching undergraduates, is listed with \input{./includes/wellesley.works} works. 

\subsection{Comparison}

No two bibliometric data sources are identical, and we observe certain differences in these two databases as well. This is also true of different snapshots of the same database over time, even for data that ostensibly refers to citations occurring several years back. Thus, variability is to be expected. 

In this section, we compare our subsample of the two databases, for the same articles and authors, where possible. Table~\ref{tab:metrics} is for the smaller sample with manually collected WoS data. Table~\ref{tab:metrics:OA:WoS} reproduces Table~\ref{tab:metrics:OA} in the main text, matched to the WoS extract, but collected at a later stage.



% Table created by stargazer v.5.2.3 by Marek Hlavac, Social Policy Institute. E-mail: marek.hlavac at gmail.com
% Date and time: Sat, Mar 16, 2024 - 12:51:42 AM
\begin{table}[!htbp] \centering 
  \caption{Publication and Author Metrics, WoS} 
  \label{tab:metrics} 
\begin{tabular}{@{\extracolsep{0.4pt}} cccc} 
\\[-1.8ex]\hline 
\hline \\[-1.8ex] 
  & Unsuccessful & Partial & Successful \\ 
\hline \\[-1.8ex] 
Avg h-index & 7.14 & 7.22 & 7.78 \\ 
Lowest h-index & 5.07 & 4.25 & 4.51 \\ 
Number of Authors & 2.14 & 2.4 & 2.58 \\ 
Citations & 4.04 & 3.6 & 5.26 \\ 
N & 14 & 52 & 45 \\ 
\hline \\[-1.8ex] 
\multicolumn{4}{l}{Notes: Articles with attempted reproduction.} \\ 
\multicolumn{4}{l}{Bibliometric data manually queried from WoS.} \\ 
\end{tabular} 
\end{table} 



% Table created by stargazer v.5.2.3 by Marek Hlavac, Social Policy Institute. E-mail: marek.hlavac at gmail.com
% Date and time: Sat, Mar 16, 2024 - 12:58:11 AM
\begin{table}[!htbp] \centering 
  \caption{Publication and Author Metrics for WoS sample} 
  \label{tab:metrics:OA:WoS} 
\begin{tabular}{@{\extracolsep{0.4pt}} cccc} 
\\[-1.8ex]\hline 
\hline \\[-1.8ex] 
  & Unsuccessful & Partial & Successful \\ 
\hline \\[-1.8ex] 
Avg h-index & 20.76 & 17.84 & 19.65 \\ 
Lowest h-index & 14.57 & 10.38 & 10.73 \\ 
Number of Authors & 2.14 & 2.42 & 2.58 \\ 
Citations & 7.07 & 8.56 & 12.04 \\ 
Highest experience & 25.79 & 23.4 & 25.27 \\ 
Institutional productivity & 26.57 & 18.12 & 23.2 \\ 
Percent of authors in US & 71.43 & 73.08 & 82.22 \\ 
N & 14 & 52 & 45 \\ 
\hline \\[-1.8ex] 
\multicolumn{4}{l}{Notes: Assessed articles matched to Web of Science database extract. Author } \\ 
\multicolumn{4}{l}{and institutional characteristics are measured 4 years after publication.} \\ 
\multicolumn{4}{l}{Institutional (cumulative) productivity measured in 10,000 publications.} \\ 
\end{tabular} 
\end{table} 


In general, the absolute level of citations and derivative h-indexes are much higher in OA. In both databases, however, the approximate relative levels are similar.
